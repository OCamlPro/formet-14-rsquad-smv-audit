
\chapter{Contract PadawanResolver}

\minitoc

\section{Overview}

In file {\tt PadawanResolver.sol}

This contract is inherited by contracts that need to deploy {\tt
  Padawan} contract and verify that an address belongs to a deployed
{\tt Padawan} contract.

\section{Variable Definitions}

\begin{itemize}
\item OK
\end{itemize}

\begin{lstlisting}[firstnumber=8]
    TvmCell _codePadawan;
\end{lstlisting}

\section{Public Method Definitions}


\subsection{Function resolvePadawan}

\begin{itemize}
\item OK
\end{itemize}

\begin{lstlisting}[firstnumber=10]
    function resolvePadawan(address owner) public view returns (address addrPadawan) {
        TvmCell state = _buildPadawanState(owner);
        uint256 hashState = tvm.hash(state);
        addrPadawan = address.makeAddrStd(0, hashState);
    }
\end{lstlisting}

\section{Internal Method Definitions}


\subsection{Function \_{}buildPadawanState}


\begin{itemize}
\item Minor issue: the state built in this function uses {\tt
  address(this)} as one of the static variables for the contract. Yet,
  this contract is bound to be inherited by different contracts (here,
  at least {\tt Demiurge} and {\tt Proposal}), i.e. computed addresses
  will be different for different contracts. Instead, the value of the
  {\tt \_deployer} variable should be made explicit to the caller, by
  passing it as an argument of the function.
\item Minor issue: this function should fail ({\tt require}) if the
  {\tt \_codePadawan} variable has not yet been initialized. A global
  boolean could be used for that, set in an internal function
  initializing both global variables.
\end{itemize}

\begin{lstlisting}[firstnumber=16]
    function _buildPadawanState(address owner) internal virtual view returns (TvmCell) {
        return tvm.buildStateInit({
            contr: Padawan,
            varInit: {_deployer: address(this), _owner: owner},
            code: _codePadawan
        });
    }
\end{lstlisting}
