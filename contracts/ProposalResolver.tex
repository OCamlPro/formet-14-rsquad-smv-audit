
\chapter{Contract ProposalResolver}

\minitoc

\section{Overview}

In file {\tt ProposalResolver.sol}

This contract is inherited by contracts that need to deploy {\tt
  Proposal} contract and verify that an address belongs to a deployed
{\tt Proposal} contract.

\section{Variable Definitions}

\begin{itemize}
\item OK
\end{itemize}

\begin{lstlisting}[firstnumber=6]
    TvmCell _codeProposal;
\end{lstlisting}

\section{Public Method Definitions}


\subsection{Function resolveProposal}

\begin{itemize}
\item OK
\end{itemize}

\begin{lstlisting}[firstnumber=8]
    function resolveProposal(string title) public view returns (address addrProposal) {
        TvmCell state = _buildProposalState(title);
        uint256 hashState = tvm.hash(state);
        addrProposal = address.makeAddrStd(0, hashState);
    }
\end{lstlisting}

\section{Internal Method Definitions}


\subsection{Function \_{}buildProposalState}

\begin{itemize}
\item Minor issue: the state built in this function uses {\tt
  address(this)} as one of the static variables for the contract. Yet,
  this contract is bound to be inherited by different contracts
  (although here, onlye {\tt Demiurge} uses it), i.e. computed
  addresses will be different for different contracts. Instead, the
  value of the {\tt \_deployer} variable should be made explicit to
  the caller, by passing it as an argument of the function.
\item Minor issue: this function should fail ({\tt require}) if the
  {\tt \_codeProposal} variable has not yet been initialized. A global
  boolean could be used for that, set in an internal function
  initializing both global variables.
\end{itemize}

\begin{lstlisting}[firstnumber=14]
    function _buildProposalState(string title) internal view returns (TvmCell) {
        return tvm.buildStateInit({
            contr: Proposal,
            varInit: {_deployer: address(this), _title: title},
            code: _codeProposal
        });
    }
\end{lstlisting}
