
\chapter{Contract Faucet}

\minitoc

\section{Overview}


In file {\tt Faucet.sol}

This contract is used to create the initial voting rights of all
users. Voting rights are stored in TIP-3 token wallets created by a
root token contract.

\subsection{Constructor}

\begin{itemize}
\item \issueMajor{No permission checks in {\tt
    Faucet.constructor}}{The constructor is called with no check on
  the caller. As a consequence, an attacker could deploy the contract
  for another user (i.e. on the address for the pubkey of the other
  user), initializing the contract with his own token wallet
  address. This attack is still limited, but it might take some time
  for the user to really understand what is happening, and the user
  will have to restart the deployment on another pubkey. Fix: check
  that the pubkey signed the constructor message}
\item Minor issue: it would be safer to use the {\tt DemiurgeStore} to
  recover the address of the token root contract.
\end{itemize}

\begin{lstlisting}[firstnumber=22]
    constructor(address addrTokenRoot, address addrTokenWallet) public {
        tvm.accept();
        _addrTokenRoot = addrTokenRoot;
        _addrTokenWallet = addrTokenWallet;
    }
\end{lstlisting}

\section{Public Method Definitions}

\subsection{Function claimTokens}

\begin{itemize}
\item Minor issue: the contract should implement ``on-bounced''
  callbacks on its token wallet to recover from sending tokens to
  not-yet-deployed token wallets.
\end{itemize}

\begin{lstlisting}[firstnumber=28]
    function claimTokens(address addrTokenWallet) external override {
        require(_balances[msg.pubkey()] != 0, Errors.INVALID_CALLER);
        tvm.accept();

        _totalDistributed = _balances[msg.pubkey()];
        ITokenWallet(_addrTokenWallet).transfer(addrTokenWallet, _balances[msg.pubkey()], 0.1 ton);

        delete _balances[msg.pubkey()];
    }
\end{lstlisting}

\subsection{Function deployWallet}

\begin{itemize}
\item \issueCritical{No limitation on {\tt Faucet.deployWallet}}{A
  malicious user owning a balance in the contract can drain the
  contract balance by sending many {\tt deployWallet} messages to the
  contract. Every message spends 0.5 ton of the balance. The owner of
  the contract has no way to block the attack, as the attack remains
  possible as long as the user does not use his balance with {\tt
    claimTokens}. Fix: the contract could remember that a deployment
  request was already done by this user.}
\end{itemize}

\begin{lstlisting}[firstnumber=49]
    function deployWallet() external override {
        require(_balances[msg.pubkey()] != 0, Errors.INVALID_CALLER);
        tvm.accept();

        ITokenRoot(_addrTokenRoot).deployEmptyWallet
            {value: 0.5 ton, flag: 1, bounce: true}
            (0, 0, msg.pubkey(), 0, 0.25 ton);
    }
\end{lstlisting}
