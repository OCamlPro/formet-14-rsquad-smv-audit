
\chapter{Contract Contest}

\minitoc

\section{Overview}


In file {\tt Contest.sol}

\section{Constructor Definitions}

\subsection{Constructor}

\begin{itemize}
\item Minor issue (readability): an integer is used as an error. Fix:
  a constant should be defined instead.
\end{itemize}

\begin{lstlisting}[firstnumber=34]
    constructor(address addrBftgRootStore, string[] tags, uint128 prizePool, uint32 underwayDuration) public {
        require(msg.sender == _deployer, 101);
        _tags = tags;
        _stage = ContestStage.New;
        _prizePool = prizePool;
        _underwayDuration = underwayDuration;
        IBftgRootStore(addrBftgRootStore).queryCode
            {value: 0.2 ton, bounce: true}
            (ContractCode.JuryGroup);
    }
\end{lstlisting}

\section{Public Method Definitions}


\subsection{OnBounce function}

\begin{itemize}
\item Minor issue: this function should check the message name being
  bounced.
\end{itemize}

\begin{lstlisting}[firstnumber=64]
    onBounce(TvmSlice) external {
        if(_tagsPendings.exists(msg.sender)) {
            delete _tagsPendings[msg.sender];
            if(_tagsPendings.empty()) {
                _changeStage(ContestStage.Underway);
            }
        }
    }
\end{lstlisting}

\subsection{Function calcRewards}

\begin{itemize}
\item \issueCritical{No stage check in {\tt Contest.calcRewards}}{
  Because this function performs no check on the sender, and no check
  on the current stage (except the one of monotonicity in {\tt
    \_changeStage}), an attacker could use it to terminate a contest
  from any stage before the {\tt Reward} stage to that stage without
  passing through previous stages. Fix: this function should check for
  a delay after the start of the voting stage.  }
\item \issueMajor{Wrong computation in {\tt Contest.calcRewards}} {The
  interpretation of ``point value'' differs in {\tt calcRewards} and
  {\tt \_calcPointValue}. Indeed, in {\tt \_calcPointValue}, the
  ``point value'' is the value of a point for the {\bf average}
  submission score, whereas {\tt calcRewards} uses it for every point
  of a submission vote, i.e. not the average. Though the computation
  in {\tt \_calcPointValue} is not the final one, this difference in
  interpretation may lead to rewards much higher than the ones
  expected.}
\end{itemize}

\begin{lstlisting}[firstnumber=175]
    function calcRewards() public {
        _calcPointValue();
        optional(uint32, Vote[]) optSubmissionVotes = _submissionVotes.min();
        while (optSubmissionVotes.hasValue()) {
            (uint32 id, Vote[] submissionVotes) = optSubmissionVotes.get();
            for(uint8 i = 0; i < submissionVotes.length; i++) {
                _rewards[_submissions[id].addrPartisipant].total += submissionVotes[i].score * _pointValue;
            }
            optSubmissionVotes = _submissionVotes.next(id);
        }
        _changeStage(ContestStage.Reward);
    }
\end{lstlisting}

\subsection{Function changeStage}

\begin{itemize}
\item \issueCritical{Missing permission checks in {\tt
    Contest.changeStage}}{No permission checks are performed in this
  function. An attacker could freely change the stage of the contest,
  and drain the message balance using {\tt tvm.accept}.}
\end{itemize}

\begin{lstlisting}[firstnumber=234]
    function changeStage(ContestStage stage) external {
        tvm.accept();
        _stage = stage;
    }
\end{lstlisting}

\subsection{Function claimPartisipantReward}

\begin{itemize}
\item Minor issue: fix spelling of {\tt participant} instead of {\tt
  partisipant}.
\end{itemize}

\begin{lstlisting}[firstnumber=197]
    function claimPartisipantReward(uint128 amount) public {
        require(_rewards.exists(msg.sender), 107);
        require(_rewards[msg.sender].total - _rewards[msg.sender].paid >= amount, 108);
        _rewards[msg.sender].paid += amount;
        msg.sender.transfer(amount, true, 1);
    }
\end{lstlisting}

\subsection{Function getMembersCallback}

\begin{itemize}
\item Minor issue (readability): an integer is used as an error. Fix:
  a constant should be defined instead.
\item Minor issue: the test {\tt member.balance $>=$ 0} is useless as
  the field is an unsigned integer {\tt uint128}.
\end{itemize}

\begin{lstlisting}[firstnumber=87]
    function getMembersCallback(mapping(address => Member) members) external override {
        require(_tagsPendings.exists(msg.sender), 102);
        delete _tagsPendings[msg.sender];
        for((, Member member): members) {
            if(member.balance >= 0) {
                _jury[member.addr] = member;
            }
        }
        if(_tagsPendings.empty()) {
            _changeStage(ContestStage.Underway);
        }
    }
\end{lstlisting}

\subsection{Function reveal}

\begin{itemize}
\item \issueCritical{Multiple revelations in {\tt Contest.reveal}}{A
  jury can reveal his votes several times, adding them several times
  in the {\tt \_submissionVotes} table. Fix: remove submission from
  {\tt \_juryHiddenVotes} everytime  they are revealed.}
\item Minor issue (gas cost): instead of failing if {\tt oldHash} and
  {\tt newHash} differ, the function should probably returns the list
  of failed couples, and keep working for correct couples.
\item Minor issue (readability): an integer is used as an error. Fix:
  a constant should be defined instead.
\end{itemize}

\begin{lstlisting}[firstnumber=155]
    function reveal(RevealVote[] revealVotes) external {
        require(_stage == ContestStage.Reveal, 104);
        require(_jury.exists(msg.sender), 105);
        for(uint8 i = 0; i < revealVotes.length; i++) {
            uint oldHash = _juryHiddenVotes[msg.sender][revealVotes[i].submissionId].hash;
            uint newHash = hashVote(revealVotes[i].submissionId, revealVotes[i].score, revealVotes[i].comment);
            require(oldHash == newHash, 106);
            _submissionVotes[revealVotes[i].submissionId].push(Vote(msg.sender, revealVotes[i].score, revealVotes[i].comment));
        }
        msg.sender.transfer(0, true, 64);
    }
\end{lstlisting}

\subsection{Function stakePartisipantReward}

\begin{itemize}
\item Minor issue (readability): an integer is used as an error. Fix:
  a constant should be defined instead.
\end{itemize}

\begin{lstlisting}[firstnumber=204]
    function stakePartisipantReward(uint128 amount, string tag, address addrJury) public {
        require(_rewards.exists(msg.sender), 107);
        require(_rewards[msg.sender].total - _rewards[msg.sender].paid >= amount, 108);
        bool isTagExists = false;
        for(uint8 i = 0; i < _tags.length; i++) {
            if(_tags[i] == tag) isTagExists = true;
        }
        require(isTagExists, 108);
        _rewards[msg.sender].paid += amount;
        IBftgRoot(_deployer).registerMemberJuryGroup
            {value: amount, bounce: true, flag: 2}
            (tag, addrJury == address(0) ? msg.sender : addrJury);
        msg.sender.transfer(0, true, 64);
    }
\end{lstlisting}

\subsection{Function submit}

\begin{itemize}
\item \issueMajor{Unbounded storage in {\tt Contest.submit}}{Anybody can call this function. An attacker could use it to increase dramatically the cost of calling the contract by storing a very big submission into the contest storage.}
\item Minor issue (readability): an integer is used as an error. Fix:
  a constant should be defined instead.
\end{itemize}

\begin{lstlisting}[firstnumber=121]
    function submit(address addrPartisipant, string forumLink, string fileLink, uint hash) external {
        require(_stage == ContestStage.Underway, 104);
        _submissions[_submissionsCounter] = (Submission(_submissionsCounter, addrPartisipant, forumLink, fileLink, hash, uint32(now)));
        _submissionsCounter += 1;
        msg.sender.transfer(0, true, 64);
    }
\end{lstlisting}

\subsection{Function updateCode}

\begin{itemize}
\item \issueCritical{No permission check in {\tt Contest.updateCode}}{
  No check is performed on the sender of this message, allowing an
  attacker to provide his own malicious implementation of {\tt
    JuryGroup} to the contract. Fix: check the sender, or check the
  code hash of the code. }
\item \issueMajor{No gas check in {\tt Contest.updateCode}}{Given that
  this function is responsible for sending {\tt getMembers} messages
  to all jury groups, it should check by {\tt require} that the
  message contains enough gas to perform these sends. Otherwise, it
  could happen that the action phase could succeed, the contract would
  remember that it was initialized, yet the transaction would be
  aborted in the sending phase and no message would actually be sent
  by lack of gas.}
\item Minor issue: the infrastructure would probably be safer if the
  expected code hash is hardcoded in the source code, and check
  through a {\tt require}
\item Minor issue: if {\tt kind} is not {\tt ContractCode.JuryGroup},
  this function will silently return without error, whereas the user
  might interpret it as successful and initialization done. Fix:
  replace the {\tt if} by a {\tt require}.
\end{itemize}

\begin{lstlisting}[firstnumber=73]
    function updateCode(ContractCode kind, TvmCell code) external override {
        if (kind == ContractCode.JuryGroup) {
            _codeJuryGroup = code;
            _passCheck(CHECK_JURY_GROUP_CODE);
        }
        _onInit();
    }
\end{lstlisting}

\subsection{Function vote}

\begin{itemize}
\item Minor issue (readability): an integer is used as an error. Fix:
  a constant should be defined instead.
\item Minor issue: maybe this function could be relaxed to allow the
  voter to change his vote
\end{itemize}

\begin{lstlisting}[firstnumber=134]
    function vote(HiddenVote[] hiddenVotes) external {
        require(_stage == ContestStage.Voting, 104);
        require(_jury.exists(msg.sender), 105);
        for(uint8 i = 0; i < hiddenVotes.length; i++) {
            if(!_juryHiddenVotes[msg.sender].exists(hiddenVotes[i].submissionId)) {
                _juryHiddenVotes[msg.sender][hiddenVotes[i].submissionId] = hiddenVotes[i];
            }
        }
        msg.sender.transfer(0, true, 64);
    }
\end{lstlisting}

\section{Internal Method Definitions}


\subsection{Function \_{}changeStage}

\begin{itemize}
\item Minor issue (readability): an integer is used as an error. Fix:
  a constant should be defined instead.
\end{itemize}

\begin{lstlisting}[firstnumber=106]
    function _changeStage(ContestStage stage) private inline returns (ContestStage) {
        require(_stage < stage, 103);
        if (stage == ContestStage.Underway) {
            _underwayEnds = uint32(now) + _underwayDuration;
        }
        _stage = stage;
    }
\end{lstlisting}
