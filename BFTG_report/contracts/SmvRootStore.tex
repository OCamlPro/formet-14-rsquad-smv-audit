
\chapter{Contract SmvRootStore}

\minitoc

\section{Overview}


In file {\tt SmvRootStore.sol}

\section{General Minor-level Remarks}

In general, the infrastructure would be safer if this contract would
be implemented in two phases:
\begin{itemize}
\item In the Initialization phase, the contract is waiting for all the
  {\tt setXXX} methods to be called to initialize all the fields. A
  bitmap can be used to keep the current initialization state. Any
  attempt to user a {\tt getXXX} method should fail.
\item In the Post-Initalization phase, the contract accepts to reply
  to {\tt getXXX} methods, but {\tt setXXX} methods are disabled.
\end{itemize}

There is also an inconsistency between the getters and setters:
getters are generic (they take a {\tt kind} as argument), whereas
setters are specific (there is a different one for every kind).

\section{Contract Inheritance}


\noindent\begin{tabular}{|l|p{5cm}|}\hline
Base & \\\hline
ISmvRootStore & \\\hline
\end{tabular}


\section{Variable Definitions}

\begin{lstlisting}[firstnumber=10]
    mapping(uint8 => address) public _addrs;
\end{lstlisting}

\begin{lstlisting}[firstnumber=11]
    mapping(uint8 => TvmCell) public _codes;
\end{lstlisting}

\section{Public Method Definitions}


\subsection{Function queryAddr}

\begin{itemize}
\item Minor issue: a {\tt require} could be added to fail if {\tt
  kind} is not a well-known kind.
\end{itemize}

\begin{lstlisting}[firstnumber=36]
    function queryAddr(ContractAddr kind) external override {
        address addr = _addrs[uint8(kind)];
        ISmvRootStoreCallback(msg.sender).updateAddr{value: 0, flag: 64, bounce: false}(kind, addr);
    }
\end{lstlisting}

\subsection{Function queryCode}

\begin{itemize}
\item Minor issue: a {\tt require} could be added to fail if {\tt
  kind} is not a well-known kind.
\end{itemize}

\begin{lstlisting}[firstnumber=31]
    function queryCode(ContractCode kind) external override {
        TvmCell code = _codes[uint8(kind)];
        ISmvRootStoreCallback(msg.sender).updateCode{value: 0, flag: 64, bounce: false}(kind, code);
    }
\end{lstlisting}

\subsection{Function setBftgRootAddr}

\begin{itemize}
\item OK
\end{itemize}

\begin{lstlisting}[firstnumber=26]
    function setBftgRootAddr(address addr) public override signed {
        require(addr != address(0));
        _addrs[uint8(ContractAddr.BftgRoot)] = addr;
    }
\end{lstlisting}

\subsection{Function setGroupCode}

\begin{itemize}
\item Minor issue: the infrastructure would probably be safer if the
  expected code hash is hardcoded in the source code, and check
  through a {\tt require}
\end{itemize}

\begin{lstlisting}[firstnumber=19]
    function setGroupCode(TvmCell code) public override signed {
        _codes[uint8(ContractCode.Group)] = code;
    }
\end{lstlisting}

\subsection{Function setPadawanCode}

\begin{itemize}
\item Minor issue: the infrastructure would probably be safer if the
  expected code hash is hardcoded in the source code, and check
  through a {\tt require}
\end{itemize}

\begin{lstlisting}[firstnumber=13]
    function setPadawanCode(TvmCell code) public override signed {
        _codes[uint8(ContractCode.Padawan)] = code;
    }
\end{lstlisting}

\subsection{Function setProposalCode}

\begin{itemize}
\item Minor issue: the infrastructure would probably be safer if the
  expected code hash is hardcoded in the source code, and check
  through a {\tt require}
\end{itemize}

\begin{lstlisting}[firstnumber=16]
    function setProposalCode(TvmCell code) public override signed {
        _codes[uint8(ContractCode.Proposal)] = code;
    }
\end{lstlisting}

\subsection{Function setProposalFactoryCode}

\begin{itemize}
\item Minor issue: the infrastructure would probably be safer if the
  expected code hash is hardcoded in the source code, and check
  through a {\tt require}
\end{itemize}

\begin{lstlisting}[firstnumber=22]
    function setProposalFactoryCode(TvmCell code) public override signed {
        _codes[uint8(ContractCode.ProposalFactory)] = code;
    }
\end{lstlisting}
