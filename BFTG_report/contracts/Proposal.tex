
\chapter{Contract Proposal}

\minitoc

\section{Overview}


In file {\tt Proposal.sol}

\section{Constructor Definitions}


\subsection{Constructor}

\begin{itemize}
\item Minor issue: there is a limitation to 16 kB for deploy
  messages. For this constructor, the deploy message contains the code
  of {\tt Proposal}, the title and the code of {\tt Padawan}. Thus, it
  might become a problem in the future. There is already a mechanism
  in the infrastructure to download codes from the {\tt
    DemiurgeStore}, this contract should take advantage of it.
\item Minor issue: the {\tt \_voteCountModel} variable is initialized
  to {\tt SoftMajority} in this constructor, but it is not used
  anywhere. Consider removing it if no future use.
\end{itemize}

\begin{lstlisting}[firstnumber=32]
    constructor(
        address client,
        string title,
        uint128 votePrice,
        uint128 voteTotal,
        address voteProvider,
        address group,
        address[] whiteList,
        string proposalType,
        TvmCell specific,
        TvmCell codePadawan
    ) public {
        require(_deployer == msg.sender);

        _client = client;

        _votePrice = votePrice;
        _voteTotal = voteTotal;
        _voteProvider = voteProvider;

        _proposalInfo.title = title;
        _proposalInfo.start = uint32(now);
        _proposalInfo.end = uint32(now + 60 * 60 * 24 * 7);
        _proposalInfo.proposalType = proposalType;
        _proposalInfo.specific = specific;
        _proposalInfo.state = ProposalState.New;
        _proposalInfo.totalVotes = voteTotal;

        _codePadawan = codePadawan;

        if(group != address(0)) {
            _getGroupMembers(group);
        } else if (!whiteList.empty()) {
            _whiteList = whiteList;
        } else  {
            _openProposal = true;
        }

        _voteCountModel = VoteCountModel.SoftMajority;
    }
\end{lstlisting}

\section{Public Method Definitions}

\subsection{Function onGetMembers}

\begin{itemize}
\item \issueCritical{No permission check on {\tt
    Proposal.onGetMembers}}{No check is performed on the sender of
  {\tt onGetMembers}. An attacker could use it to fill the {\tt
    \_whiteList} variable with malicious members.}
\end{itemize}

\begin{lstlisting}[firstnumber=220]
    function onGetMembers(string name, address[] members) public override onlyContract { name;
        _whiteList = members;
    }
\end{lstlisting}

\subsection{Function queryStatus}

\begin{itemize}
\item Minor issue: a {\tt require} should check that the message
  contains enough value to send the message.
\end{itemize}

\begin{lstlisting}[firstnumber=191]
    function queryStatus() external override {
        IPadawan(msg.sender).updateStatus
            {value: 0, flag: 64, bounce: true}
            (_proposalInfo.state);
    }
\end{lstlisting}

\subsection{Function vote}

\begin{itemize}
\item Minor issue: a {\tt require} should check that the message
  contains enough value to send back the reply;
\item Minor issue: given that the constructor initializes {\tt
  \_proposalInfo.start} to {\tt now}, it is impossible for this
  function to return the {\tt VOTING\_NOT\_STARTED} error.
\item Minor issue: the transaction could be aborted if a {\tt
  onProposalPassed} message is sent by {\tt \_finalize} (in {\tt
  \_wrapUp}), together with {\tt rejectVote} or {\tt confirmVote}
  messages, because of the flag 64.  Need to test what happens if two
  messages are sent by the same transaction, with one of them
  containing the flag 64.
\end{itemize}


\begin{lstlisting}[firstnumber=84]
    function vote(address padawanOwner, bool choice, uint128 votes) external override {
        address addrPadawan = resolvePadawan(padawanOwner);
        uint16 errorCode = 0;

        require(_openProposal || _findInWhiteList(padawanOwner), Errors.INVALID_CALLER);

        if (addrPadawan != msg.sender) {
            errorCode = Errors.NOT_AUTHORIZED_CONTRACT;
        } else if (now < _proposalInfo.start) {
            errorCode = Errors.VOTING_NOT_STARTED;
        } else if (now > _proposalInfo.end) {
            errorCode = Errors.VOTING_HAS_ENDED;
        }

        if (errorCode > 0) {
            IPadawan(msg.sender).rejectVote{value: 0, flag: 64, bounce: true}(votes, errorCode);
        } else {
            IPadawan(msg.sender).confirmVote{value: 0, flag: 64, bounce: true}(votes, _votePrice, _voteProvider);
            if (choice) {
                _proposalInfo.votesFor += votes;
            } else {
                _proposalInfo.votesAgainst += votes;
            }
        }

        _wrapUp();
    }
\end{lstlisting}

\section{Internal Method Definitions}

\subsection{Function \_{}softMajority}

\begin{itemize}
\item \issueCritical{Division by 0 in {\tt
    Proposal.\_softMajority}}{If {\tt totalVotes=1}, this function
  fails with division by 0. Fix: the function should check that {\tt
    totalVotes>1}, and add special cases for {\tt totalVotes=1} and
  {\tt totalVotes=0}}
\item Minor issue (readability): use {\tt returns (bool passed)} to
  avoid the need to define a temporary variable and to return it.
\end{itemize}

\begin{lstlisting}[firstnumber=170]
    function _softMajority(
        uint128 yes,
        uint128 no
    ) private view returns (bool) {
        bool passed = false;
        passed = yes >= 1 + (_voteTotal / 10) + (no * ((_voteTotal / 2) - (_voteTotal / 10))) / (_voteTotal / 2);
        return passed;
    }
\end{lstlisting}

\subsection{Function \_{}tryEarlyComplete}

\begin{itemize}
\item Minor issue (readability): use {\tt returns (bool completed,
  bool passed)} to avoid the need to define temporary variables and to
  return them.
\end{itemize}

\begin{lstlisting}[firstnumber=130]
    function _tryEarlyComplete(
        uint128 yes,
        uint128 no
    ) private view returns (bool, bool) {
        (bool completed, bool passed) = (false, false);
        if (yes * 2 > _voteTotal) {
            completed = true;
            passed = true;
        } else if(no * 2 >= _voteTotal) {
            completed = true;
            passed = false;
        }
        return (completed, passed);
    }
\end{lstlisting}

\subsection{Function \_{}wrapUp}

\begin{itemize}
\item Minor issue: the function could immediately check if the state
  is above {\tt Ended} to avoid recomputing again when the state
  cannot change anymore;
\item Minor issue: there is no need to call {\tt \_changeState} before
  calling {\tt \_finalize}, as {\tt \_finalize} always calls {\tt
    \_changeState} and will thus override the state written in this
  function;
\end{itemize}

\begin{lstlisting}[firstnumber=145]
    function _wrapUp() private {
        (bool completed, bool passed) = (false, false);

        if (now > _proposalInfo.end) {
            completed = true;
            passed = _calculateVotes(_proposalInfo.votesFor, _proposalInfo.votesAgainst);
        } else {
            (completed, passed) = _tryEarlyComplete(_proposalInfo.votesFor, _proposalInfo.votesAgainst);
        }

        if (completed) {
            _changeState(ProposalState.Ended);
            _finalize(passed);
        }
    }
\end{lstlisting}
