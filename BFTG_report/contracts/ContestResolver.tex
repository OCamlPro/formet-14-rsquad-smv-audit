
\chapter{Contract ContestResolver}

\minitoc

\section{Overview}


In file {\tt ContestResolver.sol}

\section{Variable Definitions}

\begin{lstlisting}[firstnumber=8]
    TvmCell _codeContest;
\end{lstlisting}

\section{Public Method Definitions}


\subsection{Function resolveContest}

\begin{itemize}
\item OK
\end{itemize}

\begin{lstlisting}[firstnumber=10]
    function resolveContest(address deployer) public view returns (address addrContest) {
        TvmCell state = _buildContestState(deployer);
        uint256 hashState = tvm.hash(state);
        addrContest = address.makeAddrStd(0, hashState);
    }
\end{lstlisting}

\section{Internal Method Definitions}


\subsection{Function \_{}buildContestState}

\begin{itemize}
\item Minor issue: this function should fail ({\tt require}) if the
  {\tt \_codeContest} variable has not yet been initialized. A global
  boolean could be used for that, set in an internal function
  initializing both global variables.
\end{itemize}

\begin{lstlisting}[firstnumber=16]
    function _buildContestState(address deployer) internal virtual view returns (TvmCell) {
        return tvm.buildStateInit({
            contr: Contest,
            varInit: {_deployer: deployer},
            code: _codeContest
        });
    }
\end{lstlisting}
